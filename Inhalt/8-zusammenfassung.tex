\chapter{Zusammenfassung und Ausblick}
\label{cha:zusammenfassung}
Im Rahmen dieser Masterarbeit wurde ein dreiteiliges Software-System
entwickelt, dass das transparente und anonyme Sammeln von Sensordaten auf
Android-Smartphones erm�glicht. Diese drei Teile sind der {\em Mobile Data
Collector}, das {\em Mobile Data Collection Framework}, sowie ein
serverseitiges Framework zum Empfang der gesammelten Sensordaten.

Kern der Arbeit war es eine vertrauensw�rdige Plattform zu schaffen, f�r die
Entwicklung von Plug-Ins zum Sammeln von Sensordaten (die zwangl�ufig
personenbezogene Daten erheben), die es aber dem Benutzer erlaubt, alle
Aktivit�ten und gesammelten Daten des Plug-Ins einzusehen. Durch diese
Transparenz sollen Smartphone-Besitzer, Vertrauen in den {\em Mobile Data
Collector} erlangen und einen Teil ihrer pers�nlichen Daten an Plug-Ins
preisgeben, ohne einen direkten Mehrwert von den Plug-Ins zu erhalten.

Die Realisierung der erarbeiteten Anforderungen und Konzepte hat eine
erweiterbare Android-Anwendung hervorgebracht, die das bereits existierende
Security-System von Android durch Monitoring-Mechanismen erweitert. Der {\em
Mobile Data Collector} l�sst sich durch Sensordaten sammelnde Plug-Ins
erweitern, die mit dem {\em Mobile Data Collection Framework} entwickelt werden.
Vor der ersten Ausf�hrung eines Plug-In werden diese analysiert und der Benutzer
wird �ber Gefahren die vom Plug-In ausgehen k�nnen informiert. Der Benutzer
beh�lt hierbei, wie im Konzept beschrieben, immer die Kontrolle �ber seine
Daten. Er kann die Ausf�hrungszeit eines Plug-In bestimmen und wird �ber die
Ausf�hrung informiert, kann einsehen auf welche personenbezogenen Daten
zugegriffen wurden, sowie welche Daten an den Plug-In spezifischen Server
�bertragen werden sollen. Die Anonymit�t des Benutzers wird dadurch
sichergestellt, dass das Framework den Zugriff auf eindeutige Benutzerdaten gar
nicht erst zur Verf�gung stellt. Die bei der �bertragung verwendete eindeutige
Benutzerkennung erlaubt zwar die Zuordnung von Daten zu bereits empfangen
Daten, erlaubt aber nicht die Zuordnung zu realen Personen.

Die Evaluation hat gezeigt, dass sich mit dem {\em Mobile Data Collection
Framework} praxisrelevante Plug-Ins entwickeln lassen. So ist die Aufnahme von
Positionsdaten, die wohl wichtigste Art von Daten im Bereich der mobilen
Datenerfassung, relativ einfach m�glich. Auch lassen sich diese Daten einfach um
zus�tzliche Daten erweitern. Hierbei wurde bewusst darauf geachtet, dem
Entwickler m�glichst nicht bei der Entwicklung von Plug-Ins einzuschr�nken.

Die Evaluation zeigt aber auch die zurzeit bestehenden Monitoring-Grenzen des
{\em Mobile Data Collector}. So ist ein Monitoring der Kamera und des Mikrofons
nicht bzw. nur mit relativ gro�em Aufwand m�glich. Dieses Problem l�sst sich am
besten durch die Einf�hrung von OSGi l�sen. Hierbei bleibt aber Abzuwarten bis
Frameworks wie Dynamix den n�tigen Reifegrad erreicht haben, um diese im
produktiven Einsatz zu verwenden.

Das zuk�nftige Anwendungsgebiet des {\em Mobile Data Collector} liegen in
erster Linie im Bereich der Forschung. Hier existieren eine Vielzahl von
Projekten, die auf die Erhebung von personenbezogenen Daten angewiesen sind.
Aber auch in der Industrie ergeben sich viele Anwendungsfelder. Vor
allen Dingen kann der {\em Mobile Data Collector} ein M�glichkeit liefern, an
Daten zu gelangen, die normalerweise nur Mobilefunkunternehmen oder
Smartphone-Betriebssystemhersteller wie Google oder Apple besitzen.
