\chapter{Verwandte Arbeiten}
\label{cha:relatedwork}
Im Folgenden soll auf Arbeiten eingegangen werden, die bereits
einzelne Aspekte der zu entwickelnden Anwendung realisiert haben. 

\section{Sensorly}
Bei {\em Sensorly} handelt es sich um eine Android-Anwendung, mit der die
Netzwerkabdeckung von Mobilfunknetzen mithilfe von Android-Smartphones ermittelt
wird. Dazu l�dt man sich als Benutzer eine Gratisanwendung aus dem {\em
Android Market} herunter und installiert diese. Nach dem ersten Start sammelt
die Anwendung automatisch im Hintergrund Daten �ber die Netzabdeckung vom gerade
verbundenen Mobilfunkanbieter. Die Daten werden in regelm��igen Abst�nden an
einen zentralen Server �bermittelt, auf dem diese mit den Daten anderer Anwender
kombiniert werden. Auf der Internetseite \url{sensorly.com} werden die
gesammelten Daten auf einer Karte �bersichtlich visualisieren.

{\em Sensorly} repr�sentiert einen Anwendungsfall f�r die zu
entwickelnde Anwendung. Es ist mit {\em Sensorly} aber nicht m�glich die Daten
vor der �bertragung einzusehen oder der �bertragung zu widersprechen. {\em
Sensorly} stellt also eine klassische Anwendung zum Sammeln von Sensordaten dar.

\section{Flexible Android Permissions}
{\em \acf{flexp}} ist eine Anpassung des Android-Betriebssystems, die es
erm�glicht, die Rechte, die einer Anwendung bei der Installation gegeben werden,
nachtr�glich zu ver�ndern. Ein Benutzer hat also die M�glichkeit nachtr�glich zu bestimmen, auf
welche Ressourcen eine Anwendungen zugreifen darf und auf welche nicht. Um
m�gliche Probleme, die nach dem Rechteentzug auftreten
k�nnen, zu vermeiden, wird den Anwendungen suggeriert, dass sie Zugriff
auf die Ressourcen haben~\cite{flexp}.

Durch die Anpassung, welche {\em \acf{flexp}} erm�glicht, hat der Besitzer
bessere M�glichkeiten, zu entscheiden, auf welche Ressourcen (z.B. die Sensorik)
die Anwendung zugreifen darf. Dennoch hat der Nutzer -- wie auch bei {\em
Sensorly} -- nicht die M�glichkeit, einzusehen, was passiert, wenn ein Recht
trotzdem vergeben wurde. Die Kontrollm�glichkeiten �ber eine Anwendung steigen
zwar, eine vollst�ndige Transparenz ist aber noch immer nicht gegeben.

\section{iPhone Tracker}
Der {\em iPhone Tracker} erlaubt die Visualisierung von Daten auf einem
Desktop-Computer, die durch das Smartphone-Betriebssystem iOS auf dem Apple
iPhone erhoben werden~\cite{iphonetracker}. Bei diesen Daten handelt es sich um
detailierte Informationen �ber das Mobilfunknetz und in der Umgebung befindliche
\acs{wlan}-Zugangspunkte. Diese Daten werden laut Apple dazu verwendet, die
Lokalisierung des Benutzers zu beschleunigen. Im Gegenzug erlauben die Daten
aber die einfache Generierung von Bewegungsprofilen. Eine Einsicht der Daten
ohne spezielle Programme ist nicht m�glich~\cite{applegps}.

Der Fall des {\em iPhone Tracker} zeigt die Problematik bei fehlender
Transparenz: Dem Benutzer ist nicht bewusst, welche Daten �ber ihn gesammelt
werden. Kommt dies zutage, bedeutet das oft einen Vertrauensbruch mit dem
Benutzer. Aus diesem Grund soll die zu entwickelnde Anwendung dagegen
Transparenz als Standardfunktionalit�t bieten.