%---------------------------------------------------------------------------
% Frontpage
%---------------------------------------------------------------------------

\begin{titlepage}

\title{Entwicklung einer Android Anwendung zum transparenten und anonymen
Sammeln von Sensordaten}
\author{Malte Legenhausen}

\let\footnotesize\small
	\let\footnoterule\relax
	\null
	\vfil
	\vskip 30pt
	\begin{center}
		{\LARGE
		  {\includegraphics[width=80mm]{grafiken/Logo_Inst_Telematik_cropped.pdf}
			\\
			\vskip 20pt
			\Large Universit"at zu L"ubeck}\\
			Institut f"ur Telematik\\[2cm]
			{\Large Masterarbeit}\\ [2cm]
			Entwicklung einer Android Anwendung zum transparenten und anonymen Sammeln
			von Sensordaten \par}%
		\vskip 5em
		{\large \lineskip .75em
		\begin{tabular}[t]{c}
			{\Large von}\\[.5em]
			{\Large Malte Legenhausen}\\[7em]
			{\bf Aufgabenstellung und Betreuung:}\\[.5em]
			Prof.\ Dr.\ S.\ Fischer\\
			Dr.-Ing.\ Dennis\ Pfisterer\\
			M.\ Sc.\ Daniel\ Bimschas
		\end{tabular}
		\par}%
		\vfill 
		{\large
			L�beck, den \today
			\par}%
	\end{center}
	\par
	% thanks
	\vfil
	\null
\end{titlepage}

\cleardoublepage

% Erklaerung
\newpage
\vspace*{7cm}
\centerline{\bf Erkl"arung}

\vspace*{1cm}
Ich versichere, die vorliegende Arbeit selbstst"andig und nur unter Benutzung
der angegebenen Hilfsmittel angefertigt zu haben.

\vspace*{3cm}
L�beck, den \today 

\pagestyle{headings}

\cleardoublepage

% Kurzfassung und Abstract

\centerline{\bf Kurzfassung}
\bigskip
In der heutigen Zeit erfassen Smartphones sehr viele personenbezogene Daten
(z.~B. Standort, Benutzerverhalten) �ber ihren Besitzer. Diese Daten besitzen
einen hohen Wert f�r �ffentliche Einrichtungen und die Industrie. In vielen
F�llen werden diese Daten ohne das Wissen des Benutzers von Herstellern von
Hardware, Betriebssystem und Applikationen gesammelt und an diese �bertragen. 
Im Rahmen dieser Arbeit wurde ein Framework f�r das Smartphone-Betriebssystem
Android entwickelt welches die Erstellung von Anwendungen zum transparenten und
anonymen Sammeln personenbezogener Daten erm�glicht. Der Benutzer erh�lt die
M�glichkeit zu jedem Zeitpunkt alle gesammelten Daten einzusehen und kann selbst
bestimmen, welche Daten zur �bertragung freigegeben werden sollen und welche
nicht. Die Verwendung des Frameworks stellt sicher, dass der Benutzer die
Kontrolle �ber seine pers�nlichen Daten hat. Anwendungen auf Basis des
entwickelten Frameworks k�nnen somit von einem gest�rkten Vertrauen der Benutzer
profitieren.


%
\vskip 3cm
%

\centerline{\bf Abstract}
\bigskip
Today's smartphones collect a high amount of personalized data (e.g. position,
user behaviour) from the owner. This data has a high value for public
institutions and the industry. But in most cases this data is collected and
transmitted without the knowledge of the user to manufacturers of hardware,
operating systems and applications. This paper focusses the development of a
framework for the smartphone operating system Android which enables the creation
of applications that collect personalized data in a transparent and anonymous
way. At any point in time the user has the possibility to review all
the collected data and he can decide which data he wants to approve for
transfer and which not. The usage of the framework guaranteed that the user is in
control of his personalized data.

\cleardoublepage

% Aufgabenstellung

\section*{Aufgabenstellung}
Die \acf{boinc} erm�glicht es die ungenutzte Rechenleistung vieler tausender
Computer f�r die L�sung komplexer wissenschaftlicher Probleme zu nutzen. Dabei
installieren Benutzer auf Ihren privat oder beruflich genutzten Computern die
\acs{boinc} Software und in Zeiten, in denen die Rechner nicht ausgelastet sind,
nutzt \acs{boinc} die Rechenzeit zur L�sung wissenschaftlicher Probleme. Welchem
wissenschaftlichen Projekt man seine Rechenleistung zur Verf�gung stellt bleibt
dem Benutzer selbst �berlassen.

In Anlehnung an dieses Prinzip soll im Rahmen dieser Arbeit eine Anwendung
geschaffen werden welche es erm�glicht die auf Smartphones anfallenden
(potenziell personenbezogenen) Daten wissenschaftlichen Projekten zur Verf�gung
zu stellen. Dies k�nnen zum Beispiel die aktuelle Temperatur, der L�rmpegel der
Umgebung, die aktuelle Position, die verf�gbaren Netzwerkverbindungen oder die
Daten des Beschleunigungssensors sein. Die Erfassung und �bertragung von Daten
soll dabei m�glichst transparent f�r den Benutzer sein, sodass dieser Kenntnis
und volle Kontrolle �ber die erhobenen Daten hat. Die L�sung soll dabei die
Realisierung einer Vielzahl unterschiedlicher Datenerfassungsanwendungen
erm�glichen. Das Smartphone-Betriebssystem Android soll technische Basis der zu
entwickelnden L�sung sein.

Im Rahmen einer Anforderungsanalyse sollen zun�chst Anforderungen an Transparenz
und Anonymit�t sowie technische Anforderungen abgeleitet werden. Anschlie�end
sollen in der Konzeptionsphase m�gliche Technologien zur Umsetzung evaluiert und
eine dieser f�r die Realisierung ausgew�hlt werden. Auf Basis der gew�hlten
Technologie erfolgt dann ein Programmentwurf welcher in der
Implementierungsphase in ein ausf�hrbares Programm �berf�hrt werden soll. Anhand
zweier zu implementierender Demonstrator-Anwendungen soll abschlie�end im Rahmen
der Evaluation gezeigt werden dass die entwickelte L�sung f�r reale
Anwendungsszenarien geeignet ist.
