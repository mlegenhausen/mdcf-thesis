%---------------------------------------------------------------------------
% Frontpage
%---------------------------------------------------------------------------

\begin{titlepage}

\title{Entwicklung einer Android Anwendung zum transparenten und anonymen
Sammeln von Sensordaten}
\author{Malte Legenhausen}

\let\footnotesize\small
	\let\footnoterule\relax
	\null
	\vfil
	\vskip 30pt
	\begin{center}
		{\LARGE
		  {\includegraphics[width=80mm]{grafiken/Logo_Inst_Telematik_cropped.pdf}
			\\
			\vskip 20pt
			\Large Universit"at zu L"ubeck}\\
			Institut f"ur Telematik\\[2cm]
			{\Large Masterarbeit}\\ [2cm]
			Entwicklung einer Android Anwendung zum transparenten und anonymen Sammeln
			von Sensordaten \par}%
		\vskip 5em
		{\large \lineskip .75em
		\begin{tabular}[t]{c}
			{\Large von}\\[.5em]
			{\Large Malte Legenhausen}\\[7em]
			{\bf Aufgabenstellung und Betreuung:}\\[.5em]
			Prof.\ Dr.\ S.\ Fischer\\
			Dr.-Ing.\ Dennis\ Pfisterer\\
			M.\ Sc.\ Daniel\ Bimschas
		\end{tabular}
		\par}%
		\vfill 
		{\large
			L�beck, den \today
			\par}%
	\end{center}
	\par
	% thanks
	\vfil
	\null
\end{titlepage}

\cleardoublepage

% Erklaerung
\newpage
\vspace*{7cm}
\centerline{\bf Erkl"arung}

\vspace*{1cm}
Ich versichere, die vorliegende Arbeit selbstst"andig und nur unter Benutzung
der angegebenen Hilfsmittel angefertigt zu haben.

\vspace*{3cm}
L�beck, den \today 

\pagestyle{headings}

\cleardoublepage

% Kurzfassung und Abstract

\centerline{\bf Kurzfassung}
\bigskip
In der heutigen Zeit erfassen Smartphones sehr viele personenbezogene Daten
(z.B. Standort, Benutzerverhalten) �ber ihren Besitzer. Diese Daten besitzen
einen relativ hohen Wert f�r �ffentliche Einrichtungen und die Industrie. In den
meisten F�llen, werden diese Daten ohne das Wissen des Benutzers, aber nur an
wenige externe dritte �bertragen (z.B. Mobilfunkanbieter, Google, Apple). Die in
dieser Arbeit entwickelte L�sung, soll eine Android Anwendung darstellen, die
das transparente und anonyme Sammeln von Sensordaten erm�glicht. Die Anwendung
stellt dazu ein Erweiterungssystem bereit, das die kontrollierte Ausf�hrungs von
Plug-Ins erlaubt. Plug-Ins k�nnen von externen Einrichtung entwickelt
werden und beliebige Daten �ber den Benutzer sammeln. Der Benutzer erh�lt die
M�glichkeit zu jedem Zeitpunkt alle gesammelten Daten einzusehen und kann selbst
bestimmen welche Daten �bertragen werden sollen und welche nicht. Durch diese
Anwendung soll der Benutzer die Kontrolle �ber seine Daten wiedererlangen und es
auch anderen Instritutionen erm�glichen personenbezogene Daten zu sammeln.

%
\vskip 3cm
%

\centerline{\bf Abstract}
\bigskip
Today's smartphones collect a high amount of personalized data (e.g. position,
user behaviour) from the owner. This data has a relativly high value for public
institutions and the industry. But in most cases this data is transmitted
without the knowledge of the user only to few external thrid parties (e.g. 
provider of mobile telecom services, Google, Apple). The solution developed in
this thesis is an Android application that is able to collect personalized data
in a transparent and anonymous way. The application provides an extensible
system that allows the controlled executions of plug-ins. Plug-Ins can be
developed by external institutions that can collect arbitary data about the
user. At any point in time the user has the possibility to review all
the collected data and can decide which data is transfered and which not.
Through this application the user should get back control over his data and
enable other institutions to collect personalized data.

\cleardoublepage

% Aufgabenstellung

\section*{Aufgabenstellung}
Die Aufgabenstellung besteht in der Entwicklung einer Android Anwendug zum
transparenten und anonymen Sammeln von Sensordaten. Vorbild f�r die Anwendung
soll hierbei das \acf{boinc} sein. Die \acs{boinc}-Plattform erm�glicht die
ungenutze Rechenleistung von vielen tausend Computern �ber das Internet
verf�gbar zu machen. Dadurch lassen sich wie beim Grid-Computing komplexe
wissenschaftliche Probleme l�sen. Welchem wissenschaftlichen Projekt man seine
Rechenleistung zur Verf�gung stellt bleibt dem Benutzer selbst �berlassen. In
Analogie daran soll die in dieser Masterarbeit entwickelte Anwendung es
ebenfalls dem Benutzer erm�glichen welchem Projekt, er seine Sensordaten zur
Verf�gung stellt.

Als technische Grundvoraussetzung um selbst zu entscheiden, welchem Projekt man
seine Daten zur Verf�gung stellt, soll die Android Anwendung ein
Erweiterungssystem zur Verf�gung stellen. Erweiterungen k�nnen dabei von
beliebigen Projekten entwickelt werden. Dieses System soll die
kontrollierte und transparente Ausf�hrung von Erweiterungen erm�glichen. Die von
einer Erweiterung gesammelten Daten sollen sich dann zu einem
projektspezifischen Server �bertragen lassen. Dabei soll sicher gestellt werden
das keine Daten �bertragen werden, die einen Smartphone-Besitzer eindeutig
identifizieren. Zur Realisierung von Erweiterungen und dem Empfang von Daten auf
der Serverseite sollen entsprechende Frameworks zur Verf�gung gestellt werden.

Das Ziel soll es sein, anhand von Transparenz und Anonymit�t die Motivation zu
schaffen, eine Anwendung zu verwenden, die es �ffentlichen und privaten
Institutionen erlaubt, beliebige Sensordaten und damit verbunden, auch
pers�nliche Daten �ber einen Benutzer zu Sammeln.

Der Umfang der Arbeit soll dabei die folgenden Punkte umfassen:

\begin{itemize}
  \item Anforderungsanalyse: Extraktion alle Anforderungen an die Android
  Anwendung aus der Aufgabenstellung.
  \item Konzeption: F�r Android Plattform soll ein entsprechender
  Programmentwurf stattfinden. Hierbei sollen auch m�gliche
  verwendbare Technologien (z.B. OSGi) evaluiert werden.
  \item Implementierung: Das entworfene Konzept soll in ein reales Programm
  �berf�hrt werden. Dabei soll auch dokumentiert werden wie sich die Anwendung
  erweitern l�sst.
  \item Evaluation: Es soll anhand zweier Erweiterungen gezeigt werden, dass die
  implementierte Anwendung auch f�r praktische Anwendungsszenarien verwendbar
  ist. Die erste Erweiterung soll die Position von Smartphone besitzern
  protokollieren. Die zweite Erweiterung soll entsprechend der Position noch die
  Lautst�rke ermitteln. Die von den beiden Erweiterugen gesammelten Daten sollen
  zu Evaluationszwecken entsprechend visualisiert werden.
\end{itemize}