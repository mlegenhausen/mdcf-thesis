%---------------------------------------------------------------------------
% Frontpage
%---------------------------------------------------------------------------

\begin{titlepage}

\title{Entwicklung einer Android Anwendung zum transparenten und anonymen
Sammeln von Sensordaten}
\author{Malte Legenhausen}

\let\footnotesize\small
	\let\footnoterule\relax
	\null
	\vfil
	\vskip 30pt
	\begin{center}
		{\LARGE
		  {\includegraphics[width=80mm]{grafiken/Logo_Inst_Telematik_cropped.pdf}
			\\
			\vskip 20pt
			\Large Universit"at zu L"ubeck}\\
			Institut f"ur Telematik\\[2cm]
			{\Large Masterarbeit}\\ [2cm]
			Entwicklung einer Android Anwendung zum transparenten und anonymen Sammeln
			von Sensordaten \par}%
		\vskip 5em
		{\large \lineskip .75em
		\begin{tabular}[t]{c}
			{\Large von}\\[.5em]
			{\Large Malte Legenhausen}\\[7em]
			{\bf Aufgabenstellung und Betreuung:}\\[.5em]
			Prof.\ Dr.\ S.\ Fischer\\
			Dr.-Ing.\ Dennis\ Pfisterer\\
			M.\ Sc.\ Daniel\ Bimschas
		\end{tabular}
		\par}%
		\vfill 
		{\large
			L�beck, den \today
			\par}%
	\end{center}
	\par
	% thanks
	\vfil
	\null
\end{titlepage}

\cleardoublepage

% Erklaerung
\newpage
\vspace*{7cm}
\centerline{\bf Erkl"arung}

\vspace*{1cm}
Ich versichere, die vorliegende Arbeit selbstst"andig und nur unter Benutzung
der angegebenen Hilfsmittel angefertigt zu haben.

\vspace*{3cm}
L�beck, den \today 

\pagestyle{headings}

\cleardoublepage

% Kurzfassung und Abstract

\centerline{\bf Kurzfassung}
\bigskip
In der heutigen Zeit erfassen Smartphones sehr viele personenbezogene Daten
(z.B. Standort, Benutzerverhalten) �ber ihren Besitzer. Diese Daten besitzen
einen relativ hohen Wert f�r �ffentliche Einrichtungen und die Industrie. In den
meisten F�llen, werden diese Daten ohne das Wissen des Benutzers, aber nur an
wenige externe dritte �bertragen (z.B. Mobilfunkanbieter, Google, Apple). Die in
dieser Arbeit entwickelte L�sung, soll eine Android Anwendung darstellen, die
das transparente und anonyme Sammeln von Sensordaten erm�glicht. Die Anwendung
stellt dazu ein Erweiterungssystem bereit, das die kontrollierte Ausf�hrungs von
Plug-Ins erlaubt. Plug-Ins k�nnen von externen Einrichtung entwickelt
werden und beliebige Daten �ber den Benutzer sammeln. Der Benutzer erh�lt die
M�glichkeit zu jedem Zeitpunkt alle gesammelten Daten einzusehen und kann selbst
bestimmen welche Daten �bertragen werden sollen und welche nicht. Durch diese
Anwendung soll der Benutzer die Kontrolle �ber seine Daten wiedererlangen und es
auch anderen Instritutionen erm�glichen personenbezogene Daten zu sammeln.

%
\vskip 3cm
%

\centerline{\bf Abstract}
\bigskip
This thesis in short.

\cleardoublepage